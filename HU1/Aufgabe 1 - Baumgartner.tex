\documentclass[12pt,a4paper]{report}
\usepackage{amssymb,amsthm,amsmath,amscd}
\usepackage{latexsym}
\usepackage{enumerate}
\usepackage[german]{babel}
\usepackage{verbatim}
\usepackage[utf8]{inputenc}
\usepackage{hyperref}

%  theorem environments
\newtheorem{thm}{Satz}[section]
\newtheorem{cor}[thm]{Korollar}
\newtheorem{lem}[thm]{Lemma}
%\newtheorem{prop}[thm]{Proposition}
%\newtheorem{ax}{Axiom}

\theoremstyle{definition}
\newtheorem{defn}[thm]{Definition}
\newtheorem{rem}[thm]{Bemerkung}
\newtheorem{exa}[thm]{Beispiel}



\begin{document}
% ---------------
\textbf{Aufgabe 1:}
\\
\\
Wer ist Martin E. Hellman und was hat er mit „Netzen“ zu tun? Recherchieren Sie über
Wikipedia hinaus geben Sie alle Quellen an!
\\
\\
\\
Martin E.Hellman ist einer der Pioniere in der Kryptographie. Zusammen mit Whitfield Diffie entwickelte er eine sichere Methode einen öffentlichen Schlüssel über einen unsicheren Kanal (Netzwerk) zu übermitteln (Diffie-Hellman-Schlüsselaustausch ).
\\
\\
Quellen:\\
\url{http://www.heise.de/newsticker/meldung/Pioniere-der-Kryptographie-Turing-Award-fuer-Whitfield-Diffie-und-Martin-Hellman-3124354.html}\\
\\
\url{http://www.zdnet.de/88261858/kryptographie-pioniere-diffie-und-hellman-mit-turing-award-ausgezeichnet/}
% -------------
\end{document}
